\section*{Теоретические вопросы}
\addcontentsline{toc}{section}{Теоретические вопросы}

\subsection*{1. Базис Lisp}
\addcontentsline{toc}{subsection}{1. Базис Lisp}

Базис --- минимальный набор инструментов и структкр данных языка, который
позволяет реализовать любую поставленную задачи.

Базис языка представлен:
\begin{itemize}
    \item атомами;
    \item структурами;
    \item функциями

    \verb|atom, eq, cons, car, cdr, cond, quote, lambda, eval, label|.
\end{itemize}

\subsection*{2. Классификация функций}
\addcontentsline{toc}{subsection}{2. Классификация функций}

\begin{itemize}
    \item Чистые --- не зависят от внешних, глобальных данных, не создают
          побочных эффектов.
    \item Формы:
        \begin{itemize}
            \item могут иметь переменное количество параметров;
            \item к арнументам может применяться особая обработка.
        \end{itemize}
    \item Функционалы:
        \begin{itemize}
            \item могут принимать функцию в качестве аргумента;
            \item могут возвращать функцию.
        \end{itemize}
\end{itemize}

Классификация базисных функций:
\begin{itemize}
    \item селекторы;
    \item конструкторы;
    \item предикаты;
    \item функции сравнения.
\end{itemize}

\subsection*{3. Способы создание функций}
\addcontentsline{toc}{subsection}{3. Способы создание функций}

Функия может быть определена двумя способами. С помощью $\lambda$-выражения
\verb|(lambda (|$\lambda$\verb|-list) f)|, где $\lambda$-list --- список
формальных аргументов, а \verb|f| - тело функции, или макро-определения
\verb|(defun name |$\lambda$\verb|-выражение)|, где name --- имя определяемой
функции.

\subsection*{4. Работа функций cond, if, and/or}
\addcontentsline{toc}{subsection}{4. Работа функций cond, if, and/or}

\verb|(cond [(test [expression [...]]) [...]])|

Вычисляет выражения \verb|test| до тех пор, пока одно из них не окажется
истинным. Если не переданы какие-либо выражения \verb|expression|, возвращается
значение \verb|test|, в противном случае выражения вычисляются по очереди и
возвращается значение последнего. Если ни одно тестовое выражение не оказалось
истинным или не указано вообще, возвращается \verb|nil|.

\verb|(if test then [else])|

Вычисляет значение \verb|test|. Если оно истинно, вычисляет и возвращает
значение выражения \verb|then|, в противном случае вычисляет и возвращает
значение \verb|else| или возвращает \verb|nil|, если оно отсутствует.

\verb|(and [expression [...]])|

Вычисляет выражения по порядку. Если значение одного из них \verb|nil|,
дальнейшее вычисление прерывается и возвращается nil. Если все выражения
истинны, возвращается значение последнего. Если выражения не заданы, возвращает
\verb|T|.

\verb|(or [expression [...]])|

Вычисляет выражения друг за другом до тех пор, пока одно из значений не
окажется истинным. В таком случае возвращается само значение, в противном
случае – \verb|nil|. Если выражения не заданы, возвращает \verb|nil|.

